\documentclass{article}
\begin{document}

\section{Local Rendering}
Each process gives the renderer a scene
which is composed of dots, lines, triangles,
and text.
Each object in the scene has a color.
The local scene is rendered onto a drawing,
which is an RGB image plus a depth buffer
(also called a $z$-buffer).
All objects are defined in 3D space, including
text which has an ``anchor point" int 3D space.

Objects are first transformed into the camera
frame, which involves scaling, rotating, and
translating their defining points.
Once in camera space, objects are clipped by
four planes defining the viewable area.
Point clipping is a binary operation: the point
can either be seen or not.
Line clipping is fairly straight forward: each
plane may cut away an invisible portion of the line, what
remains is the visible portion of the line.
Triangle clipping is more complex: the visible
portion after clipping by a plane may be a triangle
or a quadrilateral.
If it is a quadrilateral, we represent it with two
triangles and recursively clip those triangles using
the remaining planes.

We use an orthographic projection as opposed to
perspective because this is an engineering application
and avoiding perspective allows us to more easily spot
regular patterns

\end{document}
